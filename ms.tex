%# -*- coding:utf-8 -*-
%Author: Silver
%Github上源码(不定期无负责更新):https://github.com/silverriver/tex4cugb
%如果有任何bug欢迎新的PR或联系zhengyinhe1@163.com
%%%%%%%%%%%%%%%%%%%%%%%%%%%%%%%%%%%%%%%%%%%%%%

%AutoFakeBold和AutoFakeSlant是xeCJK宏包中的选项,开启伪粗体和伪斜体
%行距被设定为linespread=1.65。虽然这比学校要求(1.5倍行距)要宽,但是这是最接近word中1.5倍行距的效果
\documentclass[a4paper,UTF8,twoside,zihao=-4,no-math,linespread=1.65,AutoFakeBold=1.8,AutoFakeSlant=true]{ctexrep}
\normalsize								%字号(因为在文档类选项里面指定了zihao=-4,所以默认4号字)
\setlength{\parskip}{0ex plus 0.5ex}	%段落间的间距

%---------加载宏包----------
\usepackage{fontspec}					%设置英文字体
\setmainfont{Nimbus Roman No9 L}

\usepackage{amsmath}			%数学公式
\usepackage{amssymb}			%数学符号
\usepackage{amsthm}				%数学定理
\usepackage[top=3.3cm,bottom=3.3cm,left=3.17cm,right=3.17cm,headsep=0.7cm,headheight=0.5cm,footskip=1.25cm]{geometry}		%页边距
\usepackage{graphicx}	%图
\usepackage{booktabs}	%三线表
\usepackage{multirow}	%做表格要用到宏包
\usepackage{hyperref}	%链接宏包


\usepackage[style=base,justification=centering,font=small]{caption}	%更改标题要用到的宏包
\DeclareCaptionLabelSeparator{nocolon}{~}							%图编号与标题之间不带冒号
\captionsetup{labelsep=nocolon}										%图编号与标题之间不带冒号:
\renewcommand{\thefigure}{\thechapter-\arabic{figure}}				%设置图标题的连接符是-
\renewcommand{\thetable}{\thechapter-\arabic{table}}				%设置表标题的连接符是-
\renewcommand{\theequation}{\thechapter-\arabic{equation}}			%设置公式编号的连接符是-

\usepackage{fancyhdr}										%设置页眉页脚
\fancyhf{}\renewcommand*{\headrulewidth}{0.75bp}
\fancyhead[L]{}
\fancyhead[R]{}
\fancyhead[CE]{\small 中国地质大学(北京)硕士学位论文}		%偶数页页眉
\fancyhead[CO]{\small \leftmark}							%奇数页页眉
\fancyfoot[C]{\thepage}										%页码

%---------设置各种间距-----------
\setlength{\intextsep} {3.5ex plus 0.2ex minus 0.2ex}			%浮动体与文字之间的距离
\setlength{\textfloatsep} {3.5ex plus 0.2ex minus 0.2ex}		%浮动体与页面顶部或者底部的距离
\setlength{\floatsep} {3ex}										%两个浮动体之间的距离
\setlength{\abovecaptionskip} {1ex plus 0.2ex minus 0.2ex }		%浮动体与标题之间的距离
\setlength{\belowcaptionskip} {-1.0ex plus 0.1ex minus 0.1ex}

%---------设置章节标题格式----------
\ctexset{chapter={
	number=\arabic{chapter},
	format=\heiti\zihao{2}\centering,			%2号黑体
	beforeskip=3ex plus 0.3ex minus 0.3ex,		%与正文内容的间距
	afterskip=3ex plus 0.3ex minus 0.3ex,
	fixskip=true,
	},
  section={
	format=\heiti\zihao{3},			%3号黑体
	beforeskip=2ex plus 0.1ex minus 0.1ex,		%与正文内容的间距
	afterskip=2ex plus 0.1ex minus 0.1ex,
	},
  subsection={
	format=\heiti\zihao{4},						%4号黑体
	beforeskip=1ex plus 0.1ex minus 0.1ex,	%与正文内容的间距
	afterskip=1ex plus 0.1ex minus 0.1ex,
	},
}

%-----------图的路径和大小,方便修改--------
\newcommand{\figpath}{../图}
\newcommand{\figwidth}{13cm}

%----------文档信息-------------
\newcommand\timu{此处填入中文题目}
\newcommand\zuozhe{此处填入姓名}
\newcommand\xuehao{\textbf{2001190001}}
\newcommand\daoshi{此处填入导师}
\newcommand\daoshizhicheng{教授}
\newcommand\zhuanye{此处填入所学专业}
\newcommand\fangxiang{此处填入研究方向}
\newcommand\riqi{2019年6月}

\newcommand\yingwentimu{English Title Here}
\newcommand\yingwenzuozhe{Author}
\newcommand\yingwenzhuanye{Major}
\newcommand\yingwenfangxiang{Research area}
\newcommand\yingwendaoshi{Advisor}
\newcommand\yingwenzhicheng{Prof.}

%----------开始正文----------------
\begin{document}
\songti			%设定字体
\abovedisplayskip 2ex			%设置公式与正文的间隔
\belowdisplayskip 2ex
\abovedisplayshortskip 2ex
\belowdisplayshortskip 2ex

%# -*- coding:utf-8 -*-
%Author: Silver
%Github上源码(不定期无负责更新):https://github.com/silverriver/tex4cugb
%如果有任何bug欢迎新的PR或联系zhengyinhe1@163.com
%%%%%%%%%%%%%%%%%%%%%%%%%%%%%%%%%%%%%%%%%%%%%%

{
\cleardoublepage
\thispagestyle{empty}
\setlength{\parindent}{0em}
\songti\fontsize{15pt}{0pt}\selectfont 分类号 \hfill 密级
\begin{center}
\vskip 30pt
\fangsong\fontsize{26pt}{0pt}\selectfont 中国地质大学(北京)
\vskip 35pt
\fangsong\fontsize{42pt}{0pt}\selectfont  \textbf{硕\ \ 士\ \ 学\ \ 位\ \ 论\ \ 文}
\vskip 80pt
\heiti\fontsize{26pt}{25pt}\selectfont {\timu}
\vskip 80pt

\fangsong\fontsize{16pt}{20pt}\selectfont
\begin{tabular}{p{34mm} p{55mm}}
  学\hspace{\stretch{1}} 号 &: \xuehao  \\
  研\hspace{\stretch{1}} 究 \hspace{\stretch{1}}生 &: \zuozhe  \\
  专\hspace{\stretch{1}}业 &: \zhuanye\\
  研\hspace{\stretch{1}}究\hspace{\stretch{1}}方\hspace{\stretch{1}}向&: \fangxiang\\
  指\hspace{\stretch{1}}导\hspace{\stretch{1}}教\hspace{\stretch{1}}师&: \daoshi ~ \daoshizhicheng
\end{tabular}

\vskip 60pt
\songti\fontsize{16pt}{0pt}\selectfont \riqi
\end{center}

\clearpage
\thispagestyle{empty}
\ 
\vskip 20pt

\fontsize{16pt}{22pt}\selectfont\centering \textbf{A Dissertation Submitted to}

\textbf{China University of Geosciences for Master Degree}
\vskip 100pt
\textbf{\yingwentimu}
\vskip 130pt
\raggedright
\setlength{\leftskip}{40pt}
\textbf{Master Candidate : \yingwenzuozhe}  \\
\textbf{Major : \yingwenzhuanye}  \\
\textbf{Study Orientation : \yingwenfangxiang}\\
\textbf{Dissertation Supervisor : \yingwenzhicheng ~ \yingwendaoshi}\\
\vskip 100pt
\centering \textbf{China University of Geosciences (Beijing)}

}
		%封面(送盲审的时候不能透露姓名,可以用title_anonymous生成不带姓名信息的封面)
%# -*- coding:utf-8 -*-
%Author: Silver
%Github上源码(不定期无负责更新):https://github.com/silverriver/tex4cugb
%如果有任何bug欢迎新的PR或联系zhengyinhe1@163.com
%%%%%%%%%%%%%%%%%%%%%%%%%%%%%%%%%%%%%%%%%%%%%%

{
\clearpage
\thispagestyle{empty}
\ 
\vskip 10pt
\heiti \LARGE \centering 声~明
\vskip 15pt
\songti \normalsize \raggedright \hspace{2em}本人声明所呈交的论文是我个人在导师指导下进行的研究工作及取得的研究成果。尽我所知,除了文中特别加以标注和致谢的地方外,论文中不包含其他人已经发表或撰写过的研究成果,也不包含为获得中国地质大学(北京)或其它教育机构的学位或证书而使用过的材料。与我一同工作的同志对本研究所做的任何贡献均已在论文中作了明确的说明并表示了谢意。

\vskip 60pt
\raggedleft \fontsize{13pt}{0pt}\selectfont 签~名:\underline{\hspace{3cm}}日~期: \underline{\hspace{3cm}}

\vskip 100pt
\centering \heiti \LARGE 关于论文使用授权的说明
\vskip 10pt

\songti \normalsize \raggedright \hspace{2em}本人完全了解中国地质大学(北京)有关保留、使用学位论文的规定,即:学校有权保留送交论文的复印件,允许论文被查阅和借阅;学校可以公布论文的全部或部分内容,可以采用影印、缩印或其他复制手段保存论文。
\vskip 20pt

\hspace{2em}$\square$公开\hspace{0.5cm}$\square$保密(\underline{\hspace{1cm}}年)\hspace{1cm}\fontsize{13pt}{0pt}\selectfont \textbf{(保密的论文在解密后应遵守此规定)}

\vskip 60pt
\songti \fontsize{13pt}{0pt}\selectfont 签~名:\underline{\hspace{3cm}}导师签名:\underline{\hspace{3cm}}日~ 期:\underline{\hspace{3cm}}	    	
}		%保密声明

\pagenumbering{roman}				%设置页码格式为罗马数字
\pagestyle{plain}					%不要页眉
%# -*- coding:utf-8 -*-
%Author: Silver
%Github上源码(不定期无负责更新):https://github.com/silverriver/tex4cugb
%如果有任何bug欢迎新的PR或联系zhengyinhe1@163.com
%%%%%%%%%%%%%%%%%%%%%%%%%%%%%%%%%%%%%%%%%%%%%%

{
%摘要用section实现,所以需要临时修改一下section的标题格式
\ctexset{section={
	format=\heiti\zihao{-3}\centering,		%3号黑体
	beforeskip=0ex,							%与正文内容的间距
	afterskip=1ex,
	},
}
\section*{\zihao{-3}摘~要}

我是摘要

\noindent 关键词:我是关键词

\clearpage

\section*{\zihao{-3}Abstract}

I am the abstract

\noindent Key words: I am the key words
}
			%摘要
\tableofcontents					%目录

\cleardoublepage					%另起一页
\pagenumbering{arabic}				%设置页码格式为阿拉伯数字
\pagestyle{fancy}					%设定页眉
\chapter{我是第一章}

第一章第一章
\section{我是第一节}

第一节第一节

\section{我是第二节}

第二节第二节

			%第1章
\chapter{研究区与数据}

\section{研究区概况}
此处填入研究区概况。此处填入研究区概况。此处填入研究区概况。此处填入研究区概况。此处填入研究区概况。此处填入研究区概况。此处填入研究区概况。此处填入研究区概况。此处填入研究区概况。

\section{数据获取与处理}
此处填入数据获取与处理,此处填入数据获取与处理。此处填入数据获取与处理,此处填入数据获取与处理。此处填入数据获取与处理,此处填入数据获取与处理。此处填入数据获取与处理,此处填入数据获取与处理。

\subsection{数据1}
此处填入数据 1 简介。此处填入数据 1 简介。此处填入数据 1 简介。此处填入数据 1 简介。此处填入数据 1 简介。此处填入数据 1 简介。此处填入数据 1 简介。

\subsection{数据2}
此处填入数据 2 简介。此处填入数据 2 简介。此处填入数据 2 简介。此处填入数据 2 简介。此处填入数据 2 简介。此处填入数据 2 简介。此处填入数据 2 简介。此处填入数据 2 简介。			%第2章
%\include{chapters/chapter3}			%第3章
%\include{chapters/chapter4}			%第4章
%\include{chapters/chapter5}			%第5章
%\include{chapters/chapter6}
\chapter*{参考文献}
\addcontentsline{toc}{chapter}{参考文献}
{
\small			%5号字
\setlength\parindent{0em}						%取消首行缩进
\everypar{\hangafter=1\hangindent=2em\relax}	%悬挂缩进

手动编写参考文献
}				%手动添加参考文献,因为没有找到合适的.bst文件。如果有更好的办法欢迎补充
\chapter*{致谢}
\addcontentsline{toc}{chapter}{致谢}
此处填入致谢内容,此处填入致谢内容。此处填入致谢内容,此处填入致谢内容。此处填入致谢内容,此处填入致谢内容。此处填入致谢内容,此处填入致谢内容。此处填入致谢内容,此处填入致谢内容。此处填入致谢内容,此处填入致谢内容。此处填入致谢内容,此处填入致谢内容。此处填入致谢内容,此处填入致谢内容。此处填入致谢内容,此处填入致谢内容。此处填入致谢内容,此处填入致谢内容。

\appendix							%开始附录
\chapter{我是附录的标题}

我是附录
		%附录A
\chapter{附录}
\subsection*{个人简介}
\underline{姓名},\underline{性别},\underline{出生年份}出生于\underline{出生地}。
\subsection*{发表论文}
{
\setlength\parindent{0em}
\everypar{\hangafter=1\hangindent=2em\relax}

第一篇论文

第二篇文论文

\subsection*{参与项目}

项目1

项目2

项目3
}
				%简历
\end{document}